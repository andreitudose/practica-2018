\documentclass{report}
\usepackage{ucs}
\usepackage{listings}
\lstset{language=C}
\usepackage[utf8x]{inputenc}
\usepackage[english,romanian]{babel}
\title{{\sc Raport asupra practicii: 25.06-06.07.2018}}
\author{Andrei Tudose}
\date{\,}
\begin{document}
\maketitle

\tableofcontents

\chapter{Introducere}

Sortarea este des folosita in lucrul cu liste. Un exemplu de folosire a sortarii il reprezinta
motoarele de cautare web, care folosesc astfel de algoritmi (Google, Yahoo, MSN).
 Există diferiți algoritmi de sortare : Bubble Sort , Selective Sort , Quick Sort , Heap Sort etc.
fiecare cu avantajele si dezavantajele sale.Algoritmul de sortare prezentat in aceasta lucrare este cel Bubble Sort.

\vskip 0.5cm

Raportul asupra practicii efectuate zilnic intre datele 25.06-06.07.2018. 
\chapter{Descrierea Algoritmului Bubble Sort}
Sortarea prin metoda bulelor se considera drept una din cele mai putin efective metode de
sortare dar cu un algoritm mai putin complicat.
Ideea de baza a sortarii prin metoda bulelor este in a parcurge tabloul de la stanga spre dreapta,
fiind comparate elementele alaturate a[ i ] si a[i+1]. Daca vor fi gasite 2 elemente neordonate
valorile lor vor fi interschimbate.
 Parcurgerea tabloului de la stinga spre dreapta se va repeta atat timp cat nu vor fi intalnite
elemente neordonate. 
\chapter{Exemplu Bubble Sort}
 Sortarea crescatoare unui vector de N elemente citite de la tastatura, prin metoda bubble sort. N-ul este, de asemenea, citit de la tastatura. 


\begin{lstlisting}

#include <iostream>
using namespace std;
int main(){
int N;
cin>>N;        
int v[100]; 
int i;
for (i = 1; i <= N; i++) 
        cin>>v[i];      
int j;
int sortat; 
do{        
        sortat = 1;
        for (i = 1; i <= N-1; i++){
                if (v[i] > v[i+1]){
                        int aux = v[i]; 
                        v[i] = v[i+1]; 
                        v[i+1] = aux;
                        sortat = 0;
                }
        }
}while(sortat == 0);

for (i = 1; i <= N; i++) 
        cout<<v[i]<<" ";        
return 0;
}
\end{lstlisting}

\chapter{Activități planificate}
\begin{enumerate}
\item  Luni, 25.06.2018 \newline
Aducerea la cunoștință a obiectivelor și cerințelor practicii de producție
\item  Marți, 26.06.2018 \newline
Configurarea sistemelor software pe calculatoare. 
\item  Miercuri, 27.06.2018 \newline
Studierea modului de lucru cu Git. Interfețe grafice de lucru cu Git (SmartGit).
\item  Joi, 28.06.2018 \newline
Studierea și practicarea LaTeX
\item  Vineri, 29.06.2018  \newline
Inițierea unei lucrări (descrierea unui algoritm, a unei teme agreate cu prof. coordonator)
\item  Luni, 02.07.2018  \newline
Lucrul asupra lucrării
\item  Marți, 03.07.2018  \newline
Lucrul asupra lucrării
\item  Miercuri, 04.07.2018  \newline
Prezentarea lucrărllor
\item  Joi, 05.07.2018 \newline
Prezentarea lucrărilor
\item  Vineri, 06.07.2018  \newline
Notarea finală a activității
\end{enumerate}
\chapter{26.06.2018}
Am desfăţurat următoarele activităţi:
\begin{itemize}
\item
Am identificat sursele pentru MikTeX, Git, SmartGit și BitBucket.
\begin{itemize}
\item
Am identificat sursele pentru MikTeX, Git, SmartGit și BitBucket.
\item
Am instalat, configurat pe calculatorul de lucru aplicațiile necesare:
\begin{itemize}
\item
MikTeX
\item
SmartGit
\item
Bitbucket
\end{itemize}
\item
Am instalat, configurat pe calculatorul de lucru aplicațiile necesare:
\begin{itemize}
\item
MikTeX
\item
SmartGit
\item
Bitbucket
\end{itemize}

\end{itemize}
\end{itemize}

\chapter{25.06.2018}
Studierea obiectivelor și cerințelor față de practica de producție. Clarificarea situațiilor incerte.
\chapter{26.06.2018}
Am studiat modul de lucru cu Git și interfața grafică de lucru cu Git (SmartGit).
\chapter{27.06.2018}
Am studiat și am practicat Latex.
\chapter{28.06.2018}
Am inițiat o lucrare scrisă în Latex.
\chapter{29.06.2018}
Am continuat lucrul asupra temei alese.
\chapter{02.07.2018}
Am continuat lucrul asupra temei și am terminat .
\chapter{03.07.2018}
Am continuat lucrul asupra temei și am terminat .
Prezentarea proiectului.
\chapter{04.07.2018}
Prezentarea proiectului.
\chapter{05.07.2018}
Prezentarea proiectului.
\chapter{06.07.2018}

Notarea finală a activității.

\chapter{Concluzii}
Am invățat să lucrez cu Latex ,Git și BitBucket.
Latex este un sistem de pregătire a textelor pentru tipărire, utilizând
calculatorul.Git este un sistem revision control care rulează pe majoritatea platformelor, inclusiv Linux, POSIX, Windows și OS X.Este folosit în echipe de dezvoltare mari, în care membrii echipei acționează oarecum independent și sunt răspândiți pe o arie geografică mare.BitBucket esteun sistem de control al versiunilor distribuite care vă facilitează colaborarea cu echipa. Singura soluție Git de colaborare care scalează la scară largă.

Referinte \cite{book:1} \cite{book:25008} \cite{book:776133} \cite{book:1045183}

% aici urmeaza declaratiile care fac posibila includerea bibliografiei in format bibtex. 

\bibliography{referinte} 
\bibliographystyle{ieeetr}

\end{document}
